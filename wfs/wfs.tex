\documentclass[a4paper]{article}

\usepackage{hyperref}
\hypersetup{colorlinks,citecolor=,urlcolor=,linkcolor=}

\usepackage{natbib}
\usepackage{bibentry}
\nobibliography*

\newcommand{\bibpara}[1]{\paragraph{\cite{#1}}\bibentry{#1}\par}

\title{Wave Field Synthesis -- Some Literature}
\author{Matthias Geier}
\date{}

\begin{document}
\maketitle

\begin{abstract}
A (more or less) chronological (and very incomplete) list of references with
some (of course totally biased) comments.
In the end there is the same list sorted by author.
All this is work-in-progress.
\end{abstract}

\bibpara{berkhout1988holographic}

\bibpara{berkhout1993wfs}

\bibpara{vogel1993wfs}

\bibpara{start1997wfs}

\bibpara{verheijen1998wfs}

\bibpara{labeeuw1998optimization}

\bibpara{sonke2000variable}

\bibpara{wittek2002opsi}

\bibpara{hulsebos2004auralization}

\bibpara{springer2006stereo_displays}

WFS combined with VR. Two people are tracked with an optical tracking system.
Stereoscopic shutter glasses.

\bibpara{corteel2006equalization}

\bibpara{corteel2007directional}

\bibpara{spors2007selection}

\bibpara{ahrens2007directional}

\bibpara{melchior2008optimization}

Tracked (single) user position, source directivity, room simulation.
3 optimizations: secondary source selection, delay correction, amplitude
correction.
Pilot study: ``improved localization and sound quality for focused sources''.
2 experiments:
one with walking around a focused source and
the other with 8
plane waves to find out if amplitudes are evenly distributed(``perceived
diffuseness'').
Extension to more than one listener, ``center of gravity'', only works if
listeners are close together.

\bibpara{leslie2009installations}

Optically tracked input devices plus Wiimote.

\bibliographystyle{plainnat}
\bibliography{wfs}

\end{document}

% vim:textwidth=80
