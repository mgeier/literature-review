\documentclass[a4paper]{article}

\usepackage{natbib}
\usepackage{bibentry}
\nobibliography*

% Fix sorting for 'Lord Rayleigh':
\providecommand{\noopsort}[1]{}

\newcommand{\bibpara}[1]{\paragraph{\cite{#1}}\bibentry{#1}\par}

\title{Binaural Audio -- Some Literature}
\author{Matthias Geier}
\date{August 2013 -- ???}

\begin{document}
\maketitle

\begin{abstract}
A (more or less) chronological (and very incomplete) list of references with
some (of course totally biased) comments.
In the end there is the same list sorted by author.
All this is work-in-progress.
\end{abstract}

\bibpara{rayleigh1907sound_direction}

The famous ``Duplex Theory''.

\bibpara{thurlow1967head_movements}

\paragraph{Cite missing [1968]:} Development of dummy head technology at
Heinrich-Hertz-Institut in Berlin.

\paragraph{Cite missing [1973]:} First dummy head productions by German
broadcaster ARD, presented at IFA Berlin. First radio play ``Demolition''.

\bibpara{gardner1973median_plane}

\bibpara{shaw1974external}

Effect of pinnae is direction-dependent

\bibpara{plenge1974localization_lateralization}

externalization, ``out there''

\bibpara{butler1977spectral_cues}

\bibpara{doll1986directional_audio}

Planning a real-time implementation. Very thorough literature review.

\bibpara{wenzel1988display}

\emph{\dots\ the realtime device is not yet complete \dots}

\bibpara{wenzel1988development}

Announcement of the ``Convolvotron'', not yet real-time.

\bibpara{wenzel1990convolvotron}

Probably the first real-time system (with head-tracking) for HRIR-convolution?
The device is called ``Convolvotron'', which is undoubtedly cool.

\bibpara{moller1992fundamentals}

\bibpara{sandvad1996dynamic}

\bibpara{blauert1997spatial_hearing}

\emph{The} standard work for spatial hearing.

\section*{Web Links}

Dummy head stereo (mostly in German):\\
\url{http://www.jokan.de/kunstkopf.html}

\bibliographystyle{plainnat}
\bibliography{binaural_audio}

\end{document}

% vim:textwidth=80
